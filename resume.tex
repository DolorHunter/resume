% LaTeX file for resume 
% This file uses the resume document class (res.cls)

\documentclass{res} 
\usepackage{helvetica} % uses helvetica postscript font (download helvetica.sty)
\usepackage{newcent}   % uses new century schoolbook postscript font 
\usepackage{hyperref}  % uses hyberlinks
\hypersetup{hidelinks}
\usepackage{geometry}
\geometry{left=1cm,right=2.5cm,top=1.5cm,bottom=1cm} 
\usepackage{CJK}

\setlength{\textheight}{9.5in} % increase text height to fit on 1-page 

\begin{document} 

\name{Zixiang Wang\\[8pt]}     % the \\[12pt] adds a blank
                        % line after name  

\address{
    Syracuse, NY $|$
    (315)-965-6606 $|$
    \href{mailto:wangtzuhsiang@gmail.com}{wangtzuhsiang@gmail.com} $|$ 
    \href{https://www.linkedin.com/in/zixiangwang/}{LinkedIn} $|$ 
    \href{https://github.com/DolorHunter}{GitHub} $|$ 
    \href{https://dolorhunter.com}{dolorhunter.com}
}

\begin{resume}
\vspace{-15pt}
\hspace{-0.55in}
\noindent\rule[0.25\baselineskip]{19.36cm}{1.2pt}    
\vspace{-20pt}  

\vspace{-0.1in}
\section{EDUCATION} 
    \vspace{-0.05in}	 
    \begin{tabbing}
    \hspace{3.49in}\= \hspace{2in}\= \kill % set up two tab positions
    {\bf Syracuse University} 
        \>Syracuse, NY \` Aug 2021 - May 2023(expected)
    \end{tabbing}\vspace{-20pt}      % suppress blank line after tabbing
    Master of Computer Science      

    \vspace{-0.15in}	 
    \begin{tabbing}
    \hspace{3.7in}\= \hspace{2in}\= \kill % set up two tab positions
    {\bf Hefei University of Technology } 
        
        \hspace{-0.1in}\begin{CJK}{UTF8}{kai} {\bf 「合肥工業大學」 } \end{CJK} 
        \>P.R.China \` Sep 2017 - Jun 2021
    \end{tabbing}\vspace{-20pt}      % suppress blank line after tabbing
    Bachelor of Computer Science   

\vspace{-12pt}
\hspace{-0.55in}
\noindent\rule[0.25\baselineskip]{19.36cm}{0.5pt}    
      
\vspace{-0.2in}
\section{RELATED SKILLS}  
    \vspace{+0.15in}
    \hspace{-0.12in} 
    \begin{tabular}{l p{4.6in}}
    {\sl Programming Languages:} & Python, Java, C/C++, SQL, C\#, HTML, JavaScript, CSS, 
                        Shell, XML, Markdown, LaTeX \\ 
    \rule{0in}{0.2in}
    {\sl Frameworks \& Tools:} & Spring, Django, React, MySQL, SQLite, Linux, Git, 
                        GitHub Actions, Jenkins, Docker, ASP.NET, Qt, VPS, Nginx, 
                        MkDocs, Tensorflow, Jekyll
    \end{tabular}   

\vspace{-5pt}
\hspace{-0.55in}
\noindent\rule[0.25\baselineskip]{19.36cm}{0.5pt}    

\vspace{-0.2in}
\section{EXPERIENCE}
    \vspace{-0.05in}	
    \begin{tabbing}
    \hspace{2.6in}\= \hspace{1.5in}\= \hspace{1.6in}\= \kill % set up two tab positions
    {\bf Software Engineer Intern} \> iSoftStone Inc. \>  
                                        Spring, Java, MySQL    \` Jul 2020 - Sep 2020\\
    \end{tabbing}\vspace{-20pt}      % suppress blank line after tabbing
    % descriptions
    \vspace{-0.15in}
    \textit{Developed {\bf back-end} interface for {\bf data visualization} 
                platform serves inside the company.}
    \vspace{+0.05in}
    \begin{itemize} \itemsep 0.5pt %reduce space between items
        \item Designed and developed {\bf RESTful} API for user authority and 
                data portal module using {\bf Spring} and {\bf Java}.
        \item Created {\bf Threads} using {\bf Asynchronous Programming} to 
                improve the performance.
        \item Built the storage system with {\bf MySQL} and {\bf XML}.
        \item Implemented {\bf Navicat} to manage the database, and used {\bf Postman} for testing the API.
        \item Tracked changes with {\bf SVN} and deployed the project with {\bf Jenkins} 
                on the server.
    \end{itemize}

\vspace{+0.05in}

\vspace{-12pt}
\hspace{-0.55in}
\noindent\rule[0.25\baselineskip]{19.36cm}{0.5pt}    
        
\vspace{-0.2in}
\section{PROJECTS}

    % first project
    \vspace{-0.05in}	 
    \begin{tabbing}
    \hspace{2.391in}\= \hspace{3in}\= \kill % set up two tab positions
    \href{https://github.com/DolorHunter/AutoTBOXDataSystem}{\bf Auto TBOX Data System}  \> 
                Spring, Java, React, JavaScript, Python, MySQL \` Nov 2020 - Jun 2021 \\
    \end{tabbing}\vspace{-20pt}      % suppress blank line after tabbing \
    % descriptions
    \vspace{-0.13in}
    \textit{Developed {\bf full-stack} websites for vehicles faults {\bf data analysis} and
             {\bf visualization} for a {\bf Global 500} corporation.}
    \vspace{-0.1in}
    \begin{itemize} \itemsep 0.5pt %reduce space between items
        \item {\bf Collated} and {\bf analyzed} vehicles faults data fetch by API interface using 
                {\bf Python} and some libraries.
        \item Designed and developed {\bf RESTful} API for a total of 12 modules with 
                {\bf Spring} and {\bf Java}.
        \item Developed front-end websites using {\bf React} and {\bf JavaScript} with 
                {\bf Material-UI} as a template, which were graphically presented analyzed results 
                like trending, distribution, real-time statistics, etc.
        \item Set up {\bf MySQL} on {\bf VPS} and used {\bf Navicat} to manage the database, and 
                using {\bf Postman} for testing the API.
        \item Built a {\bf web crawler} with {\bf Python} fetching vehicles configurations for 
                a better presentation.
    \end{itemize}

    % second project
    \vspace{-0.2in}	 
    \begin{tabbing}
    \hspace{2.391in}\= \hspace{3in}\= \kill % set up two tab positions
    \href{https://github.com/DolorHunter/AirlineTicketSystem}{\bf Airline Ticket System}  \> 
                Django, Python, HTML, CSS, SQLite3 \` Aug 2020 - Nov 2020 \\
    \end{tabbing}\vspace{-20pt}      % suppress blank line after tabbing \
    % descriptions
    \vspace{-0.13in}
    \textit{Developed {\bf full-stack} websites for airlines to sell their tickets.}
    \vspace{+0.05in}
    \begin{itemize} \itemsep 0.5pt %reduce space between items
        \item Designed and developed {\bf RESTful} API for tickets, users and admin modules with 
                {\bf Django} and {\bf Python}.
        \item Developed front-end websites using {\bf HTML} and {\bf CSS}, and using post method 
                to use the API.
        \item Implemented {\bf SQLite3} and use {\bf Navicat} to manage the database.
        \item Deployed the system with {\bf Nginx} on {\bf VPS} for the presentation.
    \end{itemize}

    % third project
    \vspace{-0.2in}	 
    \begin{tabbing}
    \hspace{2.391in}\= \hspace{3in}\= \kill % set up two tab positions
    \href{https://github.com/lib-hfut/lib-hfut}{\bf lib-hfut}  \> 
                Python, MkDocs, Shell, GitHub Action \` Jan 2020 - Present \\
    \end{tabbing}\vspace{-20pt}      % suppress blank line after tabbing \
    % descriptions
    \vspace{-0.13in}
    \textit{Built the {\bf websites} with tools and {\bf continuous integration} the repository of 
            HFUT's library.}
    \vspace{+0.05in}
    \begin{itemize} \itemsep 0.5pt %reduce space between items
        \item Developed scripts with {\bf Python} to create README files for each folder (subjects). 
        \item Implemented {\bf MkDocs} commands to build the websites from README files.
        \item Deployed the library using {\bf GitHub Action} pipeline to build the latest websites 
                when new commit occurs.
        \item {\bf Collaborated} closely with other contributors, {\bf reviewed} 
                pull requests and issues consistently.
    \end{itemize}

\end{resume}
\end{document}